\documentclass[12pt, openany, letterpaper]{memoir}
\usepackage{MyStyle}
\pagestyle{empty}
\newgeometry{hmargin=0.85in}

\begin{document}

\mainmatter

\section*{\emph{When I Heard the Learn'd Astronomer}}
\paragraph{By Walt Whitman}~

\begin{verse}	
	When I heard the learn’d astronomer,\\	
	When the proofs, the figures, were ranged in columns before me,\\	
	When I was shown the charts and diagrams, to add, divide, and measure them,\\	
	When I sitting heard the astronomer where he lectured with much applause in the lecture-room,\\	
	How soon unaccountable I became tired and sick,\\	
	Till rising and gliding out I wander’d off by myself,\\	
	In the mystical moist night-air, and from time to time,\\	
	Look’d up in perfect silence at the stars.
\end{verse}

\newpage
\newgeometry{margin=1.25in}
\section*{\emph{The Waves}}
\paragraph{By Virginia Woolf}~

\vspace{1em}
\begin{minipage}{0.3\textwidth}
\begin{center}
	I see nothing.
	
	We may sink and settle\\
	on the waves.\\
	The sea will drum\\
	in my ears.
	
	The white petals\\
	will be darkened\\
	with sea water.
	
	They will float\\
	for a moment\\
	and then sink.
	
	Rolling over\\
	the waves will\\
	shoulder me under.
	
	Everything falls in a\\
	tremendous shower,\\
	dissolving me.
\end{center}
\end{minipage}

\newpage
\section*{{\fontspec{Microsoft JhengHei}李绅} \emph{(Toiling Farmers)}}
\paragraph{By {\fontspec{Microsoft JhengHei}悯农} (Li Shen)}~

{\fontspec{Microsoft JhengHei}
\begin{verse}
	锄禾日当午,\\
	汗滴禾下土。\\
	谁知盘中餐,\\
	粒粒皆辛苦。
\end{verse}
}

\vspace{2em}
\begin{verse}
	Farmers weeding at noon,\\
	Sweat down the field soon.\\
	Who knows food on a tray\\
	Thanks to their toiling day?
\end{verse}

\newpage
\section*{\emph{Птичка (A Little Bird)}}
\paragraph{By Александр Сергеевич Пушкин (Alexander Sergeyevich Pushkin)}~

\begin{verse}
	В чужбине свято наблюдаю\\
	Родной обычай старины:\\
	На волю птичку выпускаю\\
	При светлом празднике весны. 
	
	Я стал доступен утешенью;\\
	За что на бога мне роптать,\\
	Когда хоть одному творенью\\
	Я мог свободу даровать! 
\end{verse}

\vspace{2em}
\begin{verse}
	In alien lands I keep the body\\
	Of ancient native rites and things:\\
	I gladly free a little birdie\\
	At celebration of the spring.
	
	
	I'm now free for consolation,\\
	And thankful to almighty Lord:\\
	At least, to one of his creations\\
	I've given freedom in this world!
\end{verse}

\newpage
\newgeometry{vmargin=0.9in, hmargin=1.25in}
\section*{\emph{Caged Bird}}
\paragraph{By Maya Angelou}~
\begin{verse}
	A free bird leaps\\
	on the back of the wind\\
	and floats downstream\\ 
	till the current ends\\
	and dips his wing\\
	in the orange sun rays\\
	and dares to claim the sky.
	
	But a bird that stalks\\
	down his narrow cage\\
	can seldom see through\\
	his bars of rage\\
	his wings are clipped and\\
	his feet are tied\\
	so he opens his throat to sing.
	
	The caged bird sings\\
	with a fearful trill\\ 
	of things unknown\\
	but longed for still\\
	and his tune is heard\\
	on the distant hill\\
	for the caged bird\\
	sings of freedom.
	
	The free bird thinks of another breeze\\
	and the trade winds soft through the sighing trees\\
	and the fat worms waiting on a dawn bright lawn\\
	and he names the sky his own
	
	But a caged bird stands on the grave of dreams\\
	his shadow shouts on a nightmare scream\\
	his wings are clipped and his feet are tied\\
	so he opens his throat to sing.\\
	
	The caged bird sings\\
	with a fearful trill\\ 
	of things unknown\\
	but longed for still\\
	and his tune is heard\\
	on the distant hill\\
	for the caged bird\\
	sings of freedom.\\	
\end{verse}

\newpage
\newgeometry{vmargin=1in, hmargin=1.25in}
\section*{\emph{The Mortician in San Francisco}}
\paragraph{By Randall Mann}~
\begin{verse}
	This may sound queer,\\
	but in 1985 I held the delicate hands\\
	of Dan White:\\
	I prepared him for burial; by then, Harvey Milk\\
	was made monument—no, myth—by the years\\
	since he was shot.
	
	I remember when Harvey was shot:\\
	twenty, and I knew I was queer.\\
	Those were the years,\\
	Levi’s and leather jackets holding hands\\
	on Castro Street, cheering for Harvey Milk—\\
	elected on the same day as Dan White.
	
	I often wonder about Supervisor White,\\
	who fatally shot\\
	Mayor Moscone and Supervisor Milk,\\
	who was one of us, a Castro queer.\\
	May 21, 1979: a jury hands\\
	down the sentence, seven years—
	
	in truth, five years—\\
	for ex-cop, ex-fireman Dan White,\\
	for the blood on his hands;\\
	when he confessed that he had shot\\
	the mayor and the queer,\\
	a few men in blue cheered. And Harvey Milk?
	
	Why cry over spilled milk,\\
	some wondered, semi-privately, for years—\\
	it meant “one less queer.”\\
	The jurors turned to White.\\
	If just the mayor had been shot,\\
	Dan might have had trouble on his hands—
	
	but the twelve who held his life in their hands\\
	maybe didn’t mind the death of Harvey Milk;\\
	maybe, the second murder offered him a shot\\
	at serving only a few years.\\
	In the end, he committed suicide, this Dan White.\\
	And he was made presentable by a queer.
\end{verse}

\newpage
\newgeometry{margin=1.25in}
\section*{\emph{Holy Sonnets: Death, be not proud}}
\paragraph{By John Donne}~
\begin{verse}
	Death, be not proud, though some have called thee\\
	Mighty and dreadful, for thou art not so;\\
	For those whom thou think'st thou dost overthrow\\
	Die not, poor Death, nor yet canst thou kill me.\\
	From rest and sleep, which but thy pictures be,\\
	Much pleasure; then from thee much more must flow,\\
	And soonest our best men with thee do go,\\
	Rest of their bones, and soul's delivery.\\
	Thou art slave to fate, chance, kings, and desperate men,\\
	And dost with poison, war, and sickness dwell,\\
	And poppy or charms can make us sleep as well\\
	And better than thy stroke; why swell'st thou then?\\
	One short sleep past, we wake eternally\\
	And death shall be no more; Death, thou shalt die.
\end{verse}
\end{document}
