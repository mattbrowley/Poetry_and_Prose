\documentclass[12pt, openany, letterpaper]{memoir}
\usepackage{MyStyle}
\pagestyle{empty}
\newgeometry{hmargin=0.85in}

\begin{document}

\mainmatter

\section*{\emph{When I Heard the Learn'd Astronomer}}
\paragraph{By Walt Whitman}~

\begin{verse}	
	When I heard the learn’d astronomer,\\	
	When the proofs, the figures, were ranged in columns before me,\\	
	When I was shown the charts and diagrams, to add, divide, and measure them,\\	
	When I sitting heard the astronomer where he lectured with much applause in the lecture-room,\\	
	How soon unaccountable I became tired and sick,\\	
	Till rising and gliding out I wander’d off by myself,\\	
	In the mystical moist night-air, and from time to time,\\	
	Look’d up in perfect silence at the stars.
\end{verse}

\newpage
\newgeometry{margin=1.25in}
\section*{\emph{The Waves}}
\paragraph{By Virginia Woolf}~

\vspace{1em}
\begin{minipage}{0.3\textwidth}
\begin{center}
	I see nothing.
	
	We may sink and settle\\
	on the waves.\\
	The sea will drum\\
	in my ears.
	
	The white petals\\
	will be darkened\\
	with sea water.
	
	They will float\\
	for a moment\\
	and then sink.
	
	Rolling over\\
	the waves will\\
	shoulder me under.
	
	Everything falls in a\\
	tremendous shower,\\
	dissolving me.
\end{center}
\end{minipage}

\newpage
\section*{{\fontspec{Microsoft JhengHei}李绅} \emph{(Toiling Farmers)}}
\paragraph{By {\fontspec{Microsoft JhengHei}悯农} (Li Shen)}~

{\fontspec{Microsoft JhengHei}
\begin{verse}
	锄禾日当午,\\
	汗滴禾下土。\\
	谁知盘中餐,\\
	粒粒皆辛苦。
\end{verse}
}

\vspace{2em}
\begin{verse}
	Farmers weeding at noon,\\
	Sweat down the field soon.\\
	Who knows food on a tray\\
	Thanks to their toiling day?
\end{verse}

\newpage
\section*{\emph{Птичка (A Little Bird)}}
\paragraph{By Александр Сергеевич Пушкин (Alexander Sergeyevich Pushkin)}~

\begin{verse}
	В чужбине свято наблюдаю\\
	Родной обычай старины:\\
	На волю птичку выпускаю\\
	При светлом празднике весны. 
	
	Я стал доступен утешенью;\\
	За что на бога мне роптать,\\
	Когда хоть одному творенью\\
	Я мог свободу даровать! 
\end{verse}

\vspace{2em}
\begin{verse}
	In alien lands I keep the body\\
	Of ancient native rites and things:\\
	I gladly free a little birdie\\
	At celebration of the spring.
	
	
	I'm now free for consolation,\\
	And thankful to almighty Lord:\\
	At least, to one of his creations\\
	I've given freedom in this world!
\end{verse}

\newpage
\newgeometry{vmargin=0.9in, hmargin=1.25in}
\section*{\emph{Caged Bird}}
\paragraph{By Maya Angelou}~
\begin{verse}
	A free bird leaps\\
	on the back of the wind\\
	and floats downstream\\ 
	till the current ends\\
	and dips his wing\\
	in the orange sun rays\\
	and dares to claim the sky.
	
	But a bird that stalks\\
	down his narrow cage\\
	can seldom see through\\
	his bars of rage\\
	his wings are clipped and\\
	his feet are tied\\
	so he opens his throat to sing.
	
	The caged bird sings\\
	with a fearful trill\\ 
	of things unknown\\
	but longed for still\\
	and his tune is heard\\
	on the distant hill\\
	for the caged bird\\
	sings of freedom.
	
	The free bird thinks of another breeze\\
	and the trade winds soft through the sighing trees\\
	and the fat worms waiting on a dawn bright lawn\\
	and he names the sky his own
	
	But a caged bird stands on the grave of dreams\\
	his shadow shouts on a nightmare scream\\
	his wings are clipped and his feet are tied\\
	so he opens his throat to sing.\\
	
	The caged bird sings\\
	with a fearful trill\\ 
	of things unknown\\
	but longed for still\\
	and his tune is heard\\
	on the distant hill\\
	for the caged bird\\
	sings of freedom.\\	
\end{verse}

\newpage
\newgeometry{vmargin=1in, hmargin=1.25in}
\section*{\emph{The Mortician in San Francisco}}
\paragraph{By Randall Mann}~
\begin{verse}
	This may sound queer,\\
	but in 1985 I held the delicate hands\\
	of Dan White:\\
	I prepared him for burial; by then, Harvey Milk\\
	was made monument—no, myth—by the years\\
	since he was shot.
	
	I remember when Harvey was shot:\\
	twenty, and I knew I was queer.\\
	Those were the years,\\
	Levi’s and leather jackets holding hands\\
	on Castro Street, cheering for Harvey Milk—\\
	elected on the same day as Dan White.
	
	I often wonder about Supervisor White,\\
	who fatally shot\\
	Mayor Moscone and Supervisor Milk,\\
	who was one of us, a Castro queer.\\
	May 21, 1979: a jury hands\\
	down the sentence, seven years—
	
	in truth, five years—\\
	for ex-cop, ex-fireman Dan White,\\
	for the blood on his hands;\\
	when he confessed that he had shot\\
	the mayor and the queer,\\
	a few men in blue cheered. And Harvey Milk?
	
	Why cry over spilled milk,\\
	some wondered, semi-privately, for years—\\
	it meant “one less queer.”\\
	The jurors turned to White.\\
	If just the mayor had been shot,\\
	Dan might have had trouble on his hands—
	
	but the twelve who held his life in their hands\\
	maybe didn’t mind the death of Harvey Milk;\\
	maybe, the second murder offered him a shot\\
	at serving only a few years.\\
	In the end, he committed suicide, this Dan White.\\
	And he was made presentable by a queer.
\end{verse}

\newpage
\newgeometry{margin=1.25in}
\section*{\emph{Holy Sonnets: Death, be not proud}}
\paragraph{By John Donne}~
\begin{verse}
	Death, be not proud, though some have called thee\\
	Mighty and dreadful, for thou art not so;\\
	For those whom thou think'st thou dost overthrow\\
	Die not, poor Death, nor yet canst thou kill me.\\
	From rest and sleep, which but thy pictures be,\\
	Much pleasure; then from thee much more must flow,\\
	And soonest our best men with thee do go,\\
	Rest of their bones, and soul's delivery.\\
	Thou art slave to fate, chance, kings, and desperate men,\\
	And dost with poison, war, and sickness dwell,\\
	And poppy or charms can make us sleep as well\\
	And better than thy stroke; why swell'st thou then?\\
	One short sleep past, we wake eternally\\
	And death shall be no more; Death, thou shalt die.
\end{verse}

\newpage
\section*{\emph{Ozymandius}}
\paragraph{By Percy Bysshe Shelley}~
\begin{verse}
	I met a traveller from an antique land,\\
	Who said—“Two vast and trunkless legs of stone\\
	Stand in the desert. . . . Near them, on the sand,\\
	Half sunk a shattered visage lies, whose frown,\\
	And wrinkled lip, and sneer of cold command,\\
	Tell that its sculptor well those passions read\\
	Which yet survive, stamped on these lifeless things,\\
	The hand that mocked them, and the heart that fed;\\
	And on the pedestal, these words appear:\\
	My name is Ozymandias, King of Kings;\\
	Look on my Works, ye Mighty, and despair!\\
	Nothing beside remains. Round the decay\\
	Of that colossal Wreck, boundless and bare\\
	The lone and level sands stretch far away.”
\end{verse}

\newpage
\section*{\emph{If---}}
\paragraph{By Rudyard Kippling}~
\begin{verse}
	If you can keep your head when all about you\\
	\hspace{0.5em} Are losing theirs and blaming it on you,\\
	If you can trust yourself when all men doubt you,\\
	\hspace{0.5em} But make allowance for their doubting too;\\
	If you can wait and not be tired by waiting,\\
	\hspace{0.5em} Or being lied about, don’t deal in lies,\\
	Or being hated, don’t give way to hating,\\
	\hspace{0.5em} And yet don’t look too good, nor talk too wise:
	
	If you can dream—and not make dreams your master;\\
	\hspace{0.5em} If you can think—and not make thoughts your aim;\\
	If you can meet with Triumph and Disaster\\
	\hspace{0.5em} And treat those two impostors just the same;\\
	If you can bear to hear the truth you’ve spoken\\
	\hspace{0.5em} Twisted by knaves to make a trap for fools,\\
	Or watch the things you gave your life to, broken,\\
	\hspace{0.5em} And stoop and build ’em up with worn-out tools:
	
	If you can make one heap of all your winnings\\
	\hspace{0.5em} And risk it on one turn of pitch-and-toss,\\
	And lose, and start again at your beginnings\\
	\hspace{0.5em} And never breathe a word about your loss;\\
	If you can force your heart and nerve and sinew\\
	\hspace{0.5em} To serve your turn long after they are gone,\\
	And so hold on when there is nothing in you\\
	\hspace{0.5em} Except the Will which says to them: ‘Hold on!’
	
	If you can talk with crowds and keep your virtue,\\
	\hspace{0.5em} Or walk with Kings—nor lose the common touch,\\
	If neither foes nor loving friends can hurt you,\\
	\hspace{0.5em} If all men count with you, but none too much;\\
	If you can fill the unforgiving minute\\
	\hspace{0.5em} With sixty seconds’ worth of distance run,\\
	Yours is the Earth and everything that’s in it,\\
	\hspace{0.5em} And—which is more—you’ll be a Man, my son!
\end{verse}

\newpage
\section*{\emph{Who Has Seen the Wind?}}
\paragraph{By Christina Rossetti}~
\begin{verse}
	Who has seen the wind?\\
	Neither I nor you:\\
	But when the leaves hang trembling,\\
	The wind is passing through.
	
	Who has seen the wind?\\
	Neither you nor I:\\
	But when the trees bow down their heads,\\
	The wind is passing by.
\end{verse}

\newpage
\section*{\emph{{\fontspec{Microsoft JhengHei}七步詩} (The Quatrain of Seven Steps)}}
\paragraph{By {\fontspec{Microsoft JhengHei}曹植} (Cao Zhi)}~

{\fontspec{Microsoft JhengHei}
\begin{verse}
	煮豆燃豆萁\\
	漉菽以為汁\\
	萁在釜下燃\\
	豆在釜中泣\\
	本是同根生\\
	相煎何太急
\end{verse}
}

\vspace{2em}
\begin{verse}
	People burn the beanstalk to boil beans,\\
	filtering them to extract juice.\\
	The beanstalks were burnt under the cauldron,\\
	and the beans in the cauldron wailed:\\
	“We were originally grown from the same root;\\
	Why should we hound each other to death with such impatience?”
\end{verse}

\newpage
\section*{\emph{Still I Rise}}
\paragraph{By Maya Angelou}~

\vspace{1em}
\begin{minipage}[t]{0.49\linewidth}

	You may write me down in history\\
	With your bitter, twisted lies,\\
	You may trod me in the very dirt\\
	But still, like dust, I'll rise.
	
	Does my sassiness upset you?\\
	Why are you beset with gloom?\\
	’Cause I walk like I've got oil wells\\
	Pumping in my living room.
	
	Just like moons and like suns,\\
	With the certainty of tides,\\
	Just like hopes springing high,\\
	Still I'll rise.
	
	Did you want to see me broken?\\
	Bowed head and lowered eyes?\\
	Shoulders falling down like teardrops,\\
	Weakened by my soulful cries?
	
	Does my haughtiness offend you?\\
	Don't you take it awful hard\\
	’Cause I laugh like I've got gold mines\\
	Diggin’ in my own backyard.
	
	You may shoot me with your words,\\
	You may cut me with your eyes,\\
	You may kill me with your hatefulness,\\
	But still, like air, I’ll rise.	

\end{minipage}
\begin{minipage}[t]{0.49\linewidth}
	
	Does my sexiness upset you?\\
	Does it come as a surprise\\
	That I dance like I've got diamonds\\
	At the meeting of my thighs?
	
	Out of the huts of history’s shame\\
	I rise\\
	Up from a past that’s rooted in pain\\
	I rise\\
	I'm a black ocean, leaping and wide,\\
	Welling and swelling I bear in the tide.
	
	Leaving behind nights of terror and fear\\
	I rise\\
	Into a daybreak that’s wondrously clear\\
	I rise\\
	Bringing the gifts that my ancestors gave,\\
	I am the dream and the hope of the slave.\\
	I rise\\
	I rise\\
	I rise.	

\end{minipage}

\newpage
\section*{\emph{The New Colossus}}
\paragraph{By Emma Lazarus}~
\begin{verse}
	Not like the brazen giant of Greek fame,\\
	With conquering limbs astride from land to land;\\
	Here at our sea-washed, sunset gates shall stand\\
	A mighty woman with a torch, whose flame\\
	Is the imprisoned lightning, and her name\\
	Mother of Exiles. From her beacon-hand\\
	Glows world-wide welcome; her mild eyes command\\
	The air-bridged harbor that twin cities frame.\\
	“Keep, ancient lands, your storied pomp!” cries she\\
	With silent lips. “Give me your tired, your poor,\\
	Your huddled masses yearning to breathe free,\\
	The wretched refuse of your teeming shore.\\
	Send these, the homeless, tempest-tost to me,\\
	I lift my lamp beside the golden door!”	
\end{verse}

\newpage
\section*{\emph{{\fontspec{Microsoft JhengHei}游子吟} (Song of a Traveling Son)}}
\paragraph{By {\fontspec{Microsoft JhengHei}孟郊} (Meng Jiao)}~
{\fontspec{Microsoft JhengHei}
\begin{verse}	
	慈 母 手 中 线, 游 子 身 上 衣 。\\
	临 行 密 密 缝, 意 恐 迟 迟 归 。\\
	谁 言 寸 草 心, 报 得 三 春 晖 。
\end{verse}
}

\vspace{2em}
\begin{verse}
	Thread in the hands of a loving mother\\
	Turns to clothes on the traveling son.\\
	She adds stitch after tight stitch until he leaves\\
	and worries about his return.\\
	A grass blade is bathed in spring sun;\\
	how can its inch-sized heart return such love?
\end{verse}

\newpage
\section*{\emph{Sonnet 18: Shall I compare thee to a summer’s day?}}
\paragraph{By William Shakespeare}~
\begin{verse}
	Shall I compare thee to a summer’s day?\\
	Thou art more lovely and more temperate:\\
	Rough winds do shake the darling buds of May,\\
	And summer’s lease hath all too short a date;\\
	Sometime too hot the eye of heaven shines,\\
	And often is his gold complexion dimm'd;\\
	And every fair from fair sometime declines,\\
	By chance or nature’s changing course untrimm'd;\\
	But thy eternal summer shall not fade,\\
	Nor lose possession of that fair thou ow’st;\\
	Nor shall death brag thou wander’st in his shade,\\
	When in eternal lines to time thou grow’st:\\
	\hspace{0.5em} So long as men can breathe or eyes can see,\\
	\hspace{0.5em} So long lives this, and this gives life to thee.
\end{verse}

\newpage
\newgeometry{vmargin=1.2in, hmargin=1.2in}
\section*{\emph{The Cremation of Sam McGee}}
\paragraph{By Robert W. Service}~
\begin{verse}
	{\itshape
	There are strange things done in the midnight sun\\
	\hspace{0.5em} By the men who moil for gold;\\
	The Arctic trails have their secret tales\\
	\hspace{0.5em} That would make your blood run cold;\\
	The Northern Lights have seen queer sights,\\
	\hspace{0.5em} But the queerest they ever did see\\
	Was that night on the marge of Lake Lebarge\\
	\hspace{0.5em} I cremated Sam McGee.
	}
	
	Now Sam McGee was from Tennessee, where the cotton blooms and blows.\\
	Why he left his home in the South to roam 'round the Pole, God only knows.\\
	He was always cold, but the land of gold seemed to hold him like a spell;\\
	Though he'd often say in his homely way that “he'd sooner live in hell.”
	
	On a Christmas Day we were mushing our way over the Dawson trail.\\
	Talk of your cold! through the parka's fold it stabbed like a driven nail.\\
	If our eyes we'd close, then the lashes froze till sometimes we couldn't see;\\
	It wasn't much fun, but the only one to whimper was Sam McGee.
	
	And that very night, as we lay packed tight in our robes beneath the snow,\\
	And the dogs were fed, and the stars o'erhead were dancing heel and toe,\\
	He turned to me, and “Cap,” says he, “I'll cash in this trip, I guess;\\
	And if I do, I'm asking that you won't refuse my last request.”
	
	Well, he seemed so low that I couldn't say no; then he says with a sort of moan:\\
	“It's the cursèd cold, and it's got right hold till I'm chilled clean through to the bone.\\
	Yet 'tain't being dead—it's my awful dread of the icy grave that pains;\\
	So I want you to swear that, foul or fair, you'll cremate my last remains.”
	
	A pal's last need is a thing to heed, so I swore I would not fail;\\
	And we started on at the streak of dawn; but God! he looked ghastly pale.\\
	He crouched on the sleigh, and he raved all day of his home in Tennessee;\\
	And before nightfall a corpse was all that was left of Sam McGee.
	
	There wasn't a breath in that land of death, and I hurried, horror-driven,\\
	With a corpse half hid that I couldn't get rid, because of a promise given;\\
	It was lashed to the sleigh, and it seemed to say: “You may tax your brawn and brains,\\
	But you promised true, and it's up to you to cremate those last remains.”
	
	Now a promise made is a debt unpaid, and the trail has its own stern code.\\
	In the days to come, though my lips were dumb, in my heart how I cursed that load.\\
	In the long, long night, by the lone firelight, while the huskies, round in a ring,\\
	Howled out their woes to the homeless snows— O God! how I loathed the thing.
	
	And every day that quiet clay seemed to heavy and heavier grow;\\
	And on I went, though the dogs were spent and the grub was getting low;\\
	The trail was bad, and I felt half mad, but I swore I would not give in;\\
	And I'd often sing to the hateful thing, and it hearkened with a grin.
	
	Till I came to the marge of Lake Lebarge, and a derelict there lay;\\
	It was jammed in the ice, but I saw in a trice it was called the “Alice May.”\\
	And I looked at it, and I thought a bit, and I looked at my frozen chum;\\
	Then “Here,” said I, with a sudden cry, “is my cre-ma-tor-eum.”
	
	Some planks I tore from the cabin floor, and I lit the boiler fire;\\
	Some coal I found that was lying around, and I heaped the fuel higher;\\
	The flames just soared, and the furnace roared—such a blaze you seldom see;\\
	And I burrowed a hole in the glowing coal, and I stuffed in Sam McGee.
	
	Then I made a hike, for I didn't like to hear him sizzle so;\\
	And the heavens scowled, and the huskies howled, and the wind began to blow.\\
	It was icy cold, but the hot sweat rolled down my cheeks, and I don't know why;\\
	And the greasy smoke in an inky cloak went streaking down the sky.
	
	I do not know how long in the snow I wrestled with grisly fear;\\
	But the stars came out and they danced about ere again I ventured near;\\
	I was sick with dread, but I bravely said: “I'll just take a peep inside.\\
	I guess he's cooked, and it's time I looked”; ... then the door I opened wide.
	
	And there sat Sam, looking cool and calm, in the heart of the furnace roar;\\
	And he wore a smile you could see a mile, and he said: “Please close that door.\\
	It's fine in here, but I greatly fear you'll let in the cold and storm—\\
	Since I left Plumtree, down in Tennessee, it's the first time I've been warm.”
	
	{\itshape
	There are strange things done in the midnight sun\\
	\hspace{0.5em} By the men who moil for gold;\\
	The Arctic trails have their secret tales\\
	\hspace{0.5em} That would make your blood run cold;\\
	The Northern Lights have seen queer sights,\\
	\hspace{0.5em} But the queerest they ever did see\\
	Was that night on the marge of Lake Lebarge\\
	\hspace{0.5em} I cremated Sam McGee.
	}
\end{verse}

\newpage
\newgeometry{margin=1.25in}
\section*{\emph{A Litany for Survival}}
\paragraph{By Audre Lorde}~

\vspace{1em}\noindent
\begin{minipage}[t]{0.56\linewidth}
	For those of us who live at the shoreline\\
	standing upon the constant edges of decision\\
	crucial and alone\\
	for those of us who cannot indulge\\
	the passing dreams of choice\\
	who love in doorways coming and going\\
	in the hours between dawns\\
	looking inward and outward\\
	at once before and after\\
	seeking a now that can breed\\
	futures\\
	like bread in our children’s mouths\\
	so their dreams will not reflect\\
	the death of ours;
	
	For those of us\\
	who were imprinted with fear\\
	like a faint line in the center of our foreheads\\
	learning to be afraid with our mother’s milk\\
	for by this weapon\\
	this illusion of some safety to be found\\
	the heavy-footed hoped to silence us\\
	For all of us\\
	this instant and this triumph\\
	We were never meant to survive.
\end{minipage}
\begin{minipage}[t]{0.45\linewidth}	
	And when the sun rises we are afraid\\
	it might not remain\\
	when the sun sets we are afraid\\
	it might not rise in the morning\\
	when our stomachs are full we are afraid\\
	of indigestion\\
	when our stomachs are empty we are afraid\\
	we may never eat again\\
	when we are loved we are afraid\\
	love will vanish\\
	when we are alone we are afraid\\
	love will never return\\
	and when we speak we are afraid\\
	our words will not be heard\\
	nor welcomed\\
	but when we are silent\\
	we are still afraid
	
	So it is better to speak\\
	remembering\\
	we were never meant to survive.
\end{minipage}

\newpage
\section*{\emph{O Captain! My Captain!}}
\paragraph{By Walt Whitman}~
\begin{verse}
	O Captain! my Captain! our fearful trip is done,\\
	The ship has weather’d every rack, the prize we sought is won,\\
	The port is near, the bells I hear, the people all exulting,\\
	While follow eyes the steady keel, the vessel grim and daring;\\
	\hspace{7em} But O heart! heart! heart!\\
	\hspace{8em} O the bleeding drops of red,\\
	\hspace{9em} Where on the deck my Captain lies,\\
	\hspace{10em} Fallen cold and dead.
	
	O Captain! my Captain! rise up and hear the bells;\\
	Rise up—for you the flag is flung—for you the bugle trills,\\
	For you bouquets and ribbon’d wreaths—for you the shores a-crowding,\\
	For you they call, the swaying mass, their eager faces turning;\\
	\hspace{7em} Here Captain! dear father!\\
	\hspace{8em} This arm beneath your head!\\
	\hspace{9em} It is some dream that on the deck,\\
	\hspace{10em} You’ve fallen cold and dead.
	
	My Captain does not answer, his lips are pale and still,\\
	My father does not feel my arm, he has no pulse nor will,\\
	The ship is anchor’d safe and sound, its voyage closed and done,\\
	From fearful trip the victor ship comes in with object won;\\
	\hspace{7em} Exult O shores, and ring O bells!\\
	\hspace{8em} But I with mournful tread,\\
	\hspace{9em} Walk the deck my Captain lies,\\
	\hspace{10em} Fallen cold and dead.
\end{verse}

\newpage
\section*{\emph{Элегия (Elegy)}}
\paragraph{By Александр Сергеевич Пушкин (Alexander Sergeyevich Pushkin)}~
\begin{verse}
	Безумных лет угасшее веселье\\
	Мне тяжело, как смутное похмелье.\\
	Но, как вино — печаль минувших дней\\
	В моей душе чем старе, тем сильней.\\
	Мой путь уныл. Сулит мне труд и горе\\
	Грядущего волнуемое море.
	
	Но не хочу, о други, умирать;\\
	Я жить хочу, чтоб мыслить и страдать;\\
	И ведаю, мне будут наслажденья\\
	Меж горестей, забот и треволненья:\\
	Порой опять гармонией упьюсь,\\
	Над вымыслом слезами обольюсь,\\
	И может быть — на мой закат печальный\\
	Блеснет любовь улыбкою прощальной.
\end{verse}

\vspace{2em}
\begin{verse}
	The vanished joy of my crazy years\\
	Is as heavy as gloomy hang-over.\\
	But, like wine, the sorrow of past days\\
	Is stronger with time.\\
	My path is sad. The waving sea of the future\\
	Promises me only toil and sorrow.
	
	But, O my friends, I do not wish to die,\\
	I want to live – to think and suffer.\\
	I know, I’ll have some pleasures\\
	Among woes, cares and troubles.\\
	Sometimes I’ll be drunk with harmony again,\\
	Or will weep over my visions,\\
	And it’s possible, at my sorrowful decline,\\
	Love will flash with a parting smile
\end{verse}

\newpage
\section*{\emph{{\fontspec{Malgun Gothic}귀천} (Back to Heaven)}}
\paragraph{By {\fontspec{Malgun Gothic}천상병} (Cheon Sang-byeong)}~

{\fontspec{Malgun Gothic}
\begin{verse}
	나 하늘로 돌아가리라\\
	새벽빛 와 닿으면 스러지는\\
	이슬 더불어 손에 손을 잡고,
	
	나 하늘로 돌아가리라.\\
	노을빛 함께 단둘이서\\
	기슭에서 놀다가 구름 손짓하며는,
	
	나 하늘로 돌아가리라.\\
	아름다운 이 세상 소풍 끝내는 날,\\
	가서, 아름다웠더라고 말하리라...
\end{verse}
}

\vspace{2em}
\begin{verse}
	I'll go back to heaven again\\
	hand in hand with the dew\\
	that melts at the touch of the dawning day,
	
	I'll go back to heaven again.\\
	After playing on the slopes with the dusk,\\
	when the clouds beckon, 
	
	I'll go back to heaven again.\\
	On the last day of my outing to this beautiful world,\\
	I'll go back and say: it was beautiful...
\end{verse}

\newpage
\section*{\emph{My Voice}}
\paragraph{By Rafael Campo}~
\begin{verse}
	To cure myself of wanting Cuban songs,\\
	I wrote a Cuban song about the need\\
	For people to suppress their fantasies,\\
	Especially unhealthy ones. The song\\
	Began by making reference to the sea,\\
	Because the sea is like a need so great\\
	And deep it never can be swallowed. Then\\
	The song explores some common myths\\
	About the Cuban people and their folklore:\\
	The story of a little Carib boy\\
	Mistakenly abandoned to the sea;\\
	The legend of a bird who wanted song\\
	So desperately he gave up flight; a queen\\
	Whose strength was greater than a rival king’s.\\
	The song goes on about morality,\\
	And then there is a line about the sea,\\
	How deep it is, how many creatures need\\
	Its nourishment, how beautiful it is\\
	To need. The song is ending now, because\\
	I cannot bear to hear it any longer.\\
	I call this song of needful love my voice.
\end{verse}

\newpage
\section*{\emph{[i carry your heart with me(i carry it in]}}
\paragraph{By ee cummings}~
\begin{verse}
	i carry your heart with me(i carry it in\\
	my heart)i am never without it(anywhere\\
	i go you go,my dear;and whatever is done\\
	by only me is your doing,my darling)\\
	\hspace{14em} i fear\\
	no fate(for you are my fate,my sweet)i want\\
	no world(for beautiful you are my world,my true)\\
	and it’s you are whatever a moon has always meant\\
	and whatever a sun will always sing is you
	
	here is the deepest secret nobody knows\\
	(here is the root of the root and the bud of the bud\\
	and the sky of the sky of a tree called life;which grows\\
	higher than soul can hope or mind can hide)\\
	and this is the wonder that's keeping the stars apart
	
	i carry your heart(i carry it in my heart)
\end{verse}

\newpage
\section*{\emph{Slaveships}}
\paragraph{By Lucille Clifton}~
\begin{verse}
	loaded like spoons\\
	into the belly of Jesus\\
	where we lay for weeks for months\\
	in the sweat and stink\\
	of our own breathing\\
	Jesus\\
	why do you not protect us\\
	chained to the heart of the Angel\\
	where the prayers we never tell\\
	and hot and red\\
	as our bloody ankles\\
	Jesus\\
	Angel\\
	can these be men\\
	who vomit us out from ships\\
	called Jesus    Angel    Grace of God\\
	onto a heathen country\\
	Jesus\\
	Angel\\
	ever again\\
	can this tongue speak\\
	can these bones walk\\
	Grace Of God\\
	can this sin live
\end{verse}

\newpage
\section*{\emph{Pioneers}}
\paragraph{By Carol Lynn Pearson}~
\begin{verse}
	My people were Mormon pioneers.\\	
	Is the blood still good?\\	
	They stood in awe as truth\\
	Flew by like a dove\\
	And dropped a feather in the West.\\
	Where truth flies you follow\\	
	If you are a pioneer.\\
	I have searched the skies\\
	And now and then\\
	Another feather has fallen.\\
	I have packed the handcart again\\	
	Packed it with the precious things\\
	And thrown away the rest.\\
	I will sing by the fires at night\\	
	Out there on uncharted ground\\
	Where I am my own captain of tens\\	
	Where I blow the bugle\\
	Bring myself to morning prayer\\
	Map out the miles\\
	And never know when or where\\
	Or if at all I will finally say,\\
	“This is the place,”\\
	I face the plains\\
	On a good day for walking.\\
	The sun rises\\
	And the mist clears.\\
	I will be all right:\\
	My people were Mormon Pioneers.
\end{verse}

\newpage
\section*{\emph{For All We Have and Are}}
\paragraph{By Rudyard Kipling}~

\vspace{1em}
\begin{minipage}[t]{0.49\linewidth}
	{\large \hspace{2em}1914}
	
	For all we have and are,\\
	For all our children's fate,\\
	Stand up and take the war.\\
	The Hun is at the gate!\\
	Our world has passed away,\\
	In wantonness o'erthrown.\\
	There is nothing left to-day\\
	But steel and fire and stone!\\
	Though all we knew depart,\\
	The old Commandments stand:—\\
	“In courage keep your heart,\\
	In strength lift up your hand.”
	
	Once more we hear the word\\
	That sickened earth of old:—\\
	“No law except the Sword\\
	Unsheathed and uncontrolled.”\\
	Once more it knits mankind,\\
	Once more the nations go\\
	To meet and break and bind\\
	A crazed and driven foe.	
\end{minipage}
\begin{minipage}[t]{0.49\linewidth}
	Comfort, content, delight,\\
	The ages' slow-bought gain,\\
	They shrivelled in a night.\\
	Only ourselves remain\\
	To face the naked days\\
	In silent fortitude,\\
	Through perils and dismays\\
	Renewed and re-renewed.\\
	Though all we made depart,\\
	The old Commandments stand:—\\
	“In patience keep your heart,\\
	In strength lift up your hand.”
	
	No easy hope or lies\\
	Shall bring us to our goal,\\
	But iron sacrifice\\
	Of body, will, and soul.\\
	There is but one task for all—\\
	One life for each to give.\\
	What stands if Freedom fall?\\
	Who dies if England live?
\end{minipage}

\newpage
\section*{\emph{}}
\paragraph{By }~
\begin{verse}
\end{verse}
\end{document}
